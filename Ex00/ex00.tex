\documentclass{../uebungsblatt}

\usepackage{listings}
\usepackage{xcolor}
\usepackage{url}
\usepackage{changepage}
\usepackage{framed}
\usepackage[normalem]{ulem}

\lvname{Image Analysis\\Summer Semester 2015}
\autoren{Christoph Decker\\christoph.decker@iwr.uni-heidelberg.de}
\blatttitel{Exercise 0}

%\def\showsolutions{}

\begin{document}

\lstdefinestyle{customc}{
    belowcaptionskip=0pt,
    breaklines=true,
    xleftmargin=\parindent,
    language=Python,
    showstringspaces=false,
    basicstyle=\footnotesize\ttfamily,
    keywordstyle=\bfseries\color{green!40!black},
    commentstyle=\itshape\color{purple!40!black},
    identifierstyle=\color{blue},
    stringstyle=\color{orange}
}
\lstset{
    style=customc
}
\renewenvironment{leftbar}[1][\hsize]
{%
    \def\FrameCommand
    {%
        {\color{red}\vrule width 3pt}%
        \hspace{10pt}%
        \fboxsep=\FrameSep%
    }%
    \MakeFramed{\hsize#1\advance\hsize-\width\FrameRestore}%
}
{\endMakeFramed}

\begin{framed}
\textbf{Deadline: None}
\end{framed}

\section*{Computer Setup}
\subsection*{Windows}
Download \emph{Python(x,y)} via \url{http://code.google.com/p/pythonxy/wiki/Downloads}
and install it. Make sure that -- before installation -- the installer does not complain about an existing Python version on your computer.

Although \emph{Python(x,y)} comes already with a great variety of scientific \emph{Python} packages, we might have to install additional dependencies. For instance, we will make use of the \emph{vigranumpy} and \emph{scikit-learn}
packages which you can download and install from this website:

\url{http://www.lfd.uci.edu/~gohlke/pythonlibs/}

To be able to install these packages, you need the program \emph{pip} which you can find here:

\url{https://pip.pypa.io/en/latest/installing.html}

Now you can test whether your installation was successful: Find the Python(x,y) folder in the Windows start menu and launch the contained Spyder program. In the interactive interpreter on the bottom right, enter:
\begin{lstlisting}
import vigra
import numpy
import sklearn
\end{lstlisting}
If no output appears, your installation was successful.

\subsection*{Ubuntu}

Either compile everything yourself, use \emph{pip}, or just execute the following 
line to install the system libraries we need:

\begin{lstlisting}
 sudo apt-get install build-essential python-dev python-numpy python-setuptools python-scipy libatlas-dev libatlas3-base python-matplotlib python-vigra python-sklearn spyder
\end{lstlisting}

If you want to start GUI programs such as \texttt{spyder}, always start
them from the terminal; this way they inherit all necessary environment
variables that have been set in \texttt{\~{}/.bashrc}.

\subsection*{Mac}
Install Homebrew (\url{http://brew.sh}) by running this command:
\begin{lstlisting}
 ruby -e "$(curl -fsSL https://raw.github.com/mxcl/homebrew/go)"
\end{lstlisting}

If you get errors here, follow these instructions: \\
\url{http://www.moncefbelyamani.com/how-to-install-xcode-homebrew-git-rvm-ruby-on-mac/}

Then, insert the Homebrew directory at the top of your \texttt{PATH} variable by 
adding the following line at the bottom of your \texttt{\~{}/.bash\_profile} file:
\begin{lstlisting}
 export PATH=/usr/local/bin:$PATH
\end{lstlisting}

Now open a new terminal and run the following command:
\begin{lstlisting}
 brew doctor
\end{lstlisting}

As long as you do not get \texttt{Your system is ready to brew}, try to fix the warnings.

In the next step, you can install Python using Homebrew via 
\begin{lstlisting}
 brew install python
\end{lstlisting}

You can then install most python packages using \emph{pip}:
\begin{lstlisting}
 pip install numpy
 pip install scikit-learn
 pip install matplotlib
 pip install scipy
 pip install scikit-image
 pip install spyder
\end{lstlisting}

Note that it is not possible to install \emph{vigranumpy} like that. In case you need this dependency for your project, we can assist you installing it.

\emph{Troubleshooting}:
\begin{itemize}
 \item If \lstinline!pip install scipy! fails, you might have to run \lstinline!brew install gfortran! first.
 \item If \lstinline!pip install matplotlib! fails, try to do the following
  \begin{lstlisting}
   brew install freetype
   brew install libpng
   brew link --force freetype
  \end{lstlisting}
  and then reinstall \lstinline!scipy!.
\end{itemize}



\section*{The Scientific Python Ecosystem}

In this course, we are going to use the \emph{Python} programming language.
There exist many highly usable and well maintained scientific libraries for
Python. Packages that will be useful for us are:
\begin{itemize}
\item\emph{numpy} (\url{numpy.org}), provides multi-dimensional arrays
and fast numeric routines that work with these arrays.
%
\item\emph{scipy} (\url{scipy.org}), a collection of many scientific algorithms for areas such as optimization, linear algebra, integration, interpolation,
Fourier transforms, signal and image processing. Makes use of numpy arrays.
%
\item\emph{matplotlib} (\url{matplotlib.org}), a plotting library which provides
a MATLAB like interface. Check out the great gallery with many examples.
%
\item\emph{scikit-learn} (\url{scikit-learn.org}), a quickly growing collection
of machine learning algorithms, such as Support Vector Machines, Decision Trees,
Nearest Neighbor Methods and many more. Their website offers good overview
documentation and great examples, too.
%
\item\emph{scikit-image} (\url{scikit-image.org}), a collection of image
processing algorithms.
%
\item\emph{vigranumpy} (\url{hci.iwr.uni-heidelberg.de/vigra}), a
C++ library for multidimensional arrays, image processing and machine learning,
developed in our group. It exposes most functions to Python via the 
vigranumpy module.
\end{itemize}

All of these packages can be easily installed on recent Linux distributions,
such as Ubuntu.

\paragraph{Coming from MATLAB?}
Read \url{http://www.scipy.org/NumPy_for_Matlab_Users}. Indexing starts at
zero in numpy!

\paragraph{IPython}
For an interactive python shell, start \texttt{ipython} within a shell.
It supports auto-completion using the tab key, as well as showing the
documentation of functions by appending a \texttt{?} to the function name.
%Recent version of IPython have added the \emph{IPython notebook}, a
%``web-based interactive computational environment where you can combine
%code execution, text, mathematics, plots and rich media into a single
%document''. You may want to use it writing up your solutions:
%\url{ipython.org/notebook.html}.

\paragraph{Spyder}
Spyder (\url{http://code.google.com/p/spyderlib}) is a MATLAB-like IDE
for scientific Python. You can use the editor on the left or the 
interactive interpreter on the bottom right, just like in MATLAB.

\subsection*{Python}

For an introduction to Python, we recommend
\url{http://docs.python.org/tutorial}, sections 1 through 9. In the following, we give a very brief overview of the language in order to get you started quickly. 

\paragraph{Indentation}
Python uses \emph{indentation} to group statements as opposed to curly braces
in C-like languages. Always indent using 4 spaces. Also note that braces
around statements following \lstinline!if!, \lstinline!for!, etc. are omitted;
instead, a colon \lstinline!:! at the end is used.
\vspace*{-0.5em}\begin{lstlisting}
if a == 42:
    for i in range(5):
        print i
else:
    b = 5
\end{lstlisting}\vspace*{-0.5em}

\paragraph{Variables}
Python, as a scripting language, uses dynamic typing.
\vspace*{-0.5em} \begin{lstlisting}
a = 42                   #int
b1, b2 = 42.0, float(42) #float, cast to float
c = MyClass(a)           #instance of MyClass
d = None                 #special 'none' type
e = "Hello"              #string
f = [1,2,3]              #a list of values (mutable), f[0]=1
g = (1,2,3)              #a tuple of values (immutable), g[0]=1
h = {"x": 1, "y": 2}     #a dictionary h['x']=1, h['y']=2
\end{lstlisting}\vspace*{-0.5em}

\paragraph{Functions}
Functions are declared with \lstinline!def!. They take a list of required
arguments and optional keyword arguments (with default values), and may \lstinline!return! values.
\vspace*{-0.5em} \begin{lstlisting}
def some_func(arg1, arg2, kwarg1=True, kwarg2=42):
    return arg1 + arg2
\end{lstlisting}\vspace*{-0.5em}

\paragraph{Control Flow}
if, elif, else:
\vspace*{-0.5em} \begin{lstlisting}
if (a or b) and c:
    print "x"
elif b:
    print "y"
else:
    print "z"
\end{lstlisting}\vspace*{-0.5em}
A for loop:
\vspace*{-0.5em} \begin{lstlisting}
for i in range(0,10): # range [0,10)
    print i
    if i < 2:
        continue
    if i >= 7:
        break
\end{lstlisting}\vspace*{-0.5em}

\paragraph{Lists, tuples and slicing}
List, tuple or strings can be subscripted using \emph{slice notation}.
\vspace*{-0.5em} \begin{lstlisting}
s = "Hello"
print s[1:]
print s[2:4]
print s[0:-1:2]
print s[::2]
\end{lstlisting}\vspace*{-0.5em}

\paragraph{Modules}
Additional functionality can be accessed by importing \emph{modules}.
If you want to import only parts of a module, use the
\lstinline!from module import symbol! notation, where the function
can be optionally renamed using \lstinline!as!:
\vspace*{-0.5em}
\lstinputlisting[language=Python]{stuff/cos.py}
\vspace*{-1.0em}
Every python file you write can be imported from you in other files:
If file \texttt{file1.py} defines a function \texttt{function1},
you can use it from \texttt{file2.py} in the same directory via
\texttt{from file1 import function1}.

\paragraph{Files}
Write \lstinline!import! statements always at the beginning of your python
file.\\
Always put your main function in \lstinline!if __name__ == "__main__"! at
the end of the file. This way, the main code won't be executed if you import
this file from another file.

\subsection*{Numpy}

A good tutorial for numpy can be found at
\url{http://www.scipy.org/Tentative_NumPy_Tutorial}.

Import the numpy module via \lstinline!import numpy!

\paragraph{Creating and calculating with arrays}
\vspace*{0.25em}
\lstinputlisting[language=Python]{stuff/numpy_01.py}
\vspace*{-1.0em}

\paragraph{Array slicing}
Slicing works similar to the slicing of lists, tuples and strings as introduced
above. For multi-dimensional arrays, a slicing for each dimension is given
(comma separated).
Please see \url{http://docs.scipy.org/doc/numpy/reference/arrays.indexing.html}
for a comprehensive overview.
\vspace*{-1em}
\lstinputlisting[language=Python]{stuff/numpy_02.py}
\vspace*{-1.0em}

\paragraph{numpy.where}
To find the indices in an array where a certain condition matches, use
\lstinline!numpy.where!. The returned tuple of indices can be passed as an
argument to the \lstinline![]! operator of any array:
\vspace*{-0.5em}
\lstinputlisting[language=Python]{stuff/numpy_where.py}
\vspace*{-1.0em}

\subsection*{Vigra}
Import the \emph{vigranumpy} module using \lstinline!import vigra!.

The documentation can be found at 
\url{http://hci.iwr.uni-heidelberg.de/vigra/doc/vigranumpy/index.html}.

\emph{vigranumpy} may be used, among many other functions, to read in images as \lstinline!numpy!
arrays or write them out to file\footnote{If you did not manage to install \emph{vigra} on your machine, you might want to check out \emph{scikit-image} for some of those operations.}:
\vspace*{-0.5em} \begin{lstlisting}
import vigra
img = vigra.impex.readImage('./myImage.png')
img.shape
# red channel
img = img[...,0]
img.shape
vigra.impex.writeImage(img, './redChannel.png')
\end{lstlisting}\vspace*{-0.5em}

Note that many functions expect the array to be of a certain data type.
If you get errors like
\begin{tiny}
\begin{verbatim}
Boost.Python.ArgumentError: Python argument types in
    vigra.filters.gaussianSmoothing(numpy.ndarray, int)
did not match C++ signature:
    gaussianSmoothing(vigra::NumpyArray<4u, vigra::Multiband<float>, vigra::StridedArrayTag> array,
                      boost::python::api::object sigma, vigra::NumpyArray<4u, vigra::Multiband<float>, vigra::StridedArrayTag> out=None,
                      boost::python::api::object sigma_d=0.0, boost::python::api::object step_size=1.0, double window_size=0.0,
                      boost::python::api::object roi=None)
    gaussianSmoothing(vigra::NumpyArray<3u, vigra::Multiband<float>, vigra::StridedArrayTag> array, boost::python::api::object sigma,
                      vigra::NumpyArray<3u, vigra::Multiband<float>, vigra::StridedArrayTag> out=None, boost::python::api::object sigma_d=0.0,
                      boost::python::api::object step_size=1.0, double window_size=0.0, boost::python::api::object roi=None)
\end{verbatim}
\end{tiny}
you have forgotten to cast to the correct data type (note that it searches
above for a\\\verb!vigra::NumpyArray<4u, vigra::Multiband<float>, vigra::StridedArrayTag>!, which has pixel type float).
In this case, you can cast to 32-bit float using \lstinline!array.astype(numpy.float32)!.
Other vigra functions may use data type \lstinline!numpy.uint8! or \lstinline!numpy.uint32!.
Additionally, check that your array has the correct shape, in this case, the
function expects an array with \lstinline!ndim == 3! or \lstinline!ndim == 4!.

\subsection*{Matplotlib}
The recommended way to import matplotlib is:
\vspace*{-0.5em} \begin{lstlisting}
import matplotlib; matplotlib.use("Qt4Agg")
from mpl_toolkits.mplot3d import Axes3D
from matplotlib import pyplot as plot
\end{lstlisting}\vspace*{-0.5em}
See \url{http://matplotlib.org/gallery.html} to copy/paste some code.

\paragraph{Example: read and display image.}
\vspace*{1ex}
\lstinputlisting[language=Python]{stuff/read_img.py}

\subsection*{Other libraries}

There is a vast amount of other useful image processing and machine learning libraries for Python, such as \emph{OpenCV}, \emph{Scikit-learn} or \emph{PIL}. So whenever you are looking for a basic or special functionality, make sure to search for a matching library before starting to implement it yourself from scratch.

\section*{Exercises (optional, no points)}

\begin{enumerate}
\item Write a program that prints the numbers from 1 to 100. But for multiples of
three print ``Fizz'' instead of the number and for the multiples of five print
``Buzz''. For numbers which are multiples of both three and five print ``FizzBuzz''. {\tiny(\url{http://imranontech.com/2007/01/24/using-fizzbuzz-to-find-developers-who-grok-coding})}
 \item Plot $sin(x)$ for $x\in \[-1,1\]$. Label the axes and add a title. Then save the figure as \emph{png}. 
    \begin{adjustwidth}{2.5em}{0pt}
      \paragraph{Implementation hints:}
	Use \lstinline!numpy! and \lstinline!matplotlib.pyplot!.
    \end{adjustwidth}
 \item Write a function computing $n!$ and call the function from your \emph{main} function for $n=5$.
 \item Find and output the eigenvalues and eigenvectors of the following matrix:
\begin{equation}
 A = \left(\begin{array}{ccc}4 & 0 & -2\\ 2 & 5 & 4 \\ 0 & 0 & 5\end{array}\right)
\end{equation}
%solution:
%import numpy as np
%a = np.array([[4,0,-2],[2,5,4],[0,0,5]])
%ev, evec = np.linalg.eig(a)
Now concatenate the eigenvectors as columns in a matrix $P$, compute $P^{-1}$, and confirm that
$P \cdot \Lambda \cdot P^{-1} == A$, where $\Lambda$ is the diagonal matrix of the eigenvalues of $A$.
\begin{adjustwidth}{2.5em}{0pt}
      \paragraph{Implementation hints:}	
      Check out the documentation page of \lstinline!numpy.linalg! for eigenvalue computation and matrix inversion. Be aware that just typing $A*B$ in Python is not a real matrix multiplication.
\end{adjustwidth}
%solution:
%evecInv = np.linalg.inv(evec)
%b = np.dot(evec, np.dot(np.diag(ev),evecInv))
%np.all(a == b)
\end{enumerate}

\end{document}
